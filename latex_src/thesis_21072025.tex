\documentclass[12pt, a4paper]{report}
\usepackage[utf8]{inputenc}
\usepackage{amsmath}
\usepackage{graphicx}
\usepackage{geometry}
\usepackage{booktabs}
\usepackage{siunitx}
\usepackage{hyperref}
\usepackage{caption}
\usepackage{subcaption}
\usepackage{longtable}
\usepackage{amssymb}
\usepackage{amsfonts}
\usepackage{fancyhdr}
\usepackage{fontenc}
\usepackage{lmodern}

\geometry{a4paper, margin=1in}
\hypersetup{
    colorlinks=true,
    linkcolor=blue,
    filecolor=magenta,      
    urlcolor=cyan,
}

\pagestyle{fancy}
\fancyhf{}
\fancyhead[L]{\leftmark}
\fancyfoot[C]{\thepage}
\renewcommand{\headrulewidth}{0.4pt}
\renewcommand{\footrulewidth}{0.4pt}

\begin{document}

\title{Data-Driven Reduced-Order Modeling for Parametric Flow Reconstruction in Complex Aerospace Systems: A Contribution to the CFD Vision 2030}
\author{A. M. de Carvalho}
\date{\today}
\maketitle

\begin{abstract}
This dissertation presents a comprehensive investigation into the development and application of data-driven, machine learning-based reduced-order models (ML-ROMs) for the rapid and accurate reconstruction of complex aerodynamic flows. Positioned within the strategic framework of the NASA "CFD Vision 2030 Study," this work directly addresses critical technology gaps that currently impede the aerospace design process, including the prediction of turbulent separated flows, the bottleneck of geometry and grid generation for parametric studies, and the extraction of actionable knowledge from large-scale simulation datasets. Two distinct and challenging aerospace applications are investigated to demonstrate the robustness and versatility of the proposed methodologies.

The first study introduces a novel, fully differentiable ML-ROM framework for reconstructing two-dimensional supersonic nozzle flows characterized by strong shock wave-boundary layer interactions. A key contribution is the development of a hybrid loss function that combines errors in both the low-dimensional latent space and the reconstructed high-dimensional physical space. This reprojection-based loss acts as a physical regularizer, compelling the model to preserve high-gradient features, such as shock waves, with greater fidelity. Systematic hyperparameter optimization reveals that well-tuned shallow Artificial Neural Networks (ANNs) exhibit superior performance and robustness compared to deeper architectures and Gaussian Process (GP) models, particularly in data-scarce and noisy regimes.

The second study tackles the formidable challenge of parametric geometric variation in turbomachinery by developing a novel computational pipeline for the NASA Rotor 37 axial compressor. A mesh morphing technique, based on harmonic mapping and structured interpolation, is introduced to create a consistent topological data representation from a set of geometrically varied, unstructured computational fluid dynamics (CFD) snapshots. This enables the application of Proper Orthogonal Decomposition (POD) for dimensionality reduction. The resulting framework is the first of its kind to demonstrate the simultaneous prediction of both surface aerodynamic fields (pressure and temperature) and the underlying three-dimensional blade geometry from a set of abstract design parameters. The model achieves exceptional accuracy and a computational speed-up of over four orders of magnitude, presenting a foundational technology for next-generation multidisciplinary design analysis and optimization (MDAO) by circumventing the traditional meshing bottleneck.

Collectively, these contributions advance the state-of-the-art by demonstrating methodologies that enhance the physical fidelity, geometric versatility, and computational efficiency of surrogate models, thereby providing a tangible pathway toward achieving the revolutionary capabilities envisioned for computational aerosciences in the coming decade.
\end{abstract}

\tableofcontents

\chapter{Introduction}

\section{The Imperative for a Revolution in Computational Aerosciences}

The advent and maturation of Computational Fluid Dynamics (CFD) have fundamentally reshaped the aerospace design process over the past several decades. Its application has led to significant reductions in the time and cost associated with physical testing, such as wind tunnel campaigns and engine rig tests, while providing unparalleled insight into complex flow phenomena.[1] However, the aerospace CFD community now stands at a critical juncture. Despite exponential growth in high-performance computing (HPC) power, the practical capabilities of CFD have seen a period of stagnation. The "CFD Vision 2030 Study," a comprehensive report commissioned by NASA, identifies this issue as a primary impediment to future innovation in aerospace vehicle design.[1]

A central finding of the study is that the aerospace industry's widespread reliance on Reynolds-Averaged Navier-Stokes (RANS) methods has confined the reliable use of simulation to a narrow, but important, region of the flight envelope, typically attached-flow conditions at or near the design point.[1] The inability of current methods to reliably and accurately predict turbulent flows with significant regions of separation remains a grand challenge. This deficiency severely limits the design of next-generation systems operating at the edges of the performance envelope, such as aircraft at high-lift conditions, engines under transient or off-design operation, and space vehicles during atmospheric entry.[1] While higher-fidelity methods like Large Eddy Simulation (LES) and Direct Numerical Simulation (DNS) exist, their prohibitive computational cost renders them impractical for routine use in the iterative design cycles common to industry, creating a significant "capability gap" that new methodologies must address.[1]

Furthermore, the CFD workflow itself is plagued by persistent bottlenecks that hinder productivity and limit the scope of exploration. The Vision 2030 study highlights geometry preparation and mesh generation as one of the most significant impediments, often consuming a disproportionate amount of engineering time and effort.[1] This is particularly acute in the context of parametric studies and design optimization, where numerous geometric variations must be evaluated. Concurrently, the sheer volume of data produced by large-scale simulations presents another challenge: the extraction of actionable knowledge and physical insight from terabytes of information remains a largely manual and inefficient process. These factors collectively signal an urgent need for a paradigm shift—a move away from incremental improvements and toward revolutionary new algorithms and methodologies that can unlock the full potential of simulation-based engineering.

\section{Data-Driven Surrogate Modeling as a Pathway to 2030 Capabilities}

In response to the challenges articulated in the CFD Vision 2030 Study, data-driven surrogate modeling, particularly in the form of Machine Learning-based Reduced-Order Models (ML-ROMs), has emerged as a powerful and strategic enabling technology. ML-ROMs directly address the calls for "Knowledge Extraction" and "Revolutionary Algorithmic Improvements" by providing a framework to distill the essential physics from computationally expensive, high-fidelity simulations into compact, efficient, and predictive mathematical models.[1]

The fundamental principle of this approach is to bifurcate the computational effort into two distinct phases: an offline training stage and an online inference stage. During the offline stage, a curated set of high-fidelity CFD simulations, known as snapshots, is generated to sample a predefined design space of operational or geometric parameters. This dataset is then used to train a machine learning model to learn the mapping between the input parameters and the high-dimensional flow field solution. While this training phase can be computationally intensive, it is a one-time investment. Once trained, the resulting surrogate model can be deployed in the online phase to perform near-real-time predictions for new, unseen parameter combinations at a fraction of the cost of the original CFD solver.[1, 1] This capability unlocks the potential for rapid design space exploration, robust uncertainty quantification, and large-scale multidisciplinary design optimization (MDAO) loops that are currently intractable with conventional high-fidelity methods.

\section{Statement of Research: Gaps, Objectives, and Core Contributions}

This dissertation aims to advance the state-of-the-art in ML-ROMs by developing and validating novel frameworks that address specific, critical gaps in the existing literature and contribute directly to the goals of the CFD Vision 2030 roadmap. The research is motivated by the identification of several key deficiencies in current surrogate modeling practices:

\begin{enumerate}
    \item \textbf{The Scarcity of Parametric Flow Field Reconstruction:} A significant portion of the existing literature on surrogate modeling focuses on predicting low-dimensional, integrated quantities of interest, such as lift and drag coefficients. While valuable, this approach discards the vast majority of the information contained within the full flow field. There remains a significant need for robust methodologies that can accurately reconstruct the entire high-dimensional flow field (e.g., pressure, temperature, velocity) parametrically, especially in complex flow regimes involving nonlinear phenomena like shock waves and turbulence.[2, 3]
    \item \textbf{The Under-Exploration of Compressors:} While ROMs have been applied to various turbomachinery problems, the academic literature shows a strong focus on turbine applications or aeroelastic phenomena such as flutter.[4, 5, 6] There is a notable scarcity of parametric studies dedicated to the complex, three-dimensional aerodynamics of axial compressors, which are critical components in modern propulsion and power generation systems. The development of predictive models for compressors represents a challenging and underexplored frontier for ML-ROMs.[7, 8, 9]
    \item \textbf{The Novelty of Integrated Geometry and Flow Prediction:} Existing surrogate models almost universally assume a fixed computational domain or require a separate, often manual, process to generate the geometry and mesh for each new design parameter. A truly transformative step towards automated MDAO would be a model capable of simultaneously predicting both the aerodynamic fields and the underlying physical geometry from a set of abstract design variables. Such models are exceptionally rare in the literature and represent a critical research gap.[1]
\end{enumerate}

To address these gaps, this dissertation presents two comprehensive studies, each culminating in a core technical contribution:

\begin{itemize}
    \item \textbf{For the Nozzle Study:} The development of a robust and generalizable ML-ROM framework is presented, validated on a challenging benchmark case of supersonic flow in a conical nozzle featuring significant shock wave-boundary layer interaction (SWBLI). The primary innovation of this work is the introduction of a novel, fully differentiable hybrid loss function. This loss function incorporates a physical-space reprojection term, which penalizes the error in the fully reconstructed flow field, in addition to the traditional error in the low-dimensional latent space. This formulation acts as a powerful physical regularizer, dramatically improving the model's ability to capture high-gradient, nonlinear flow features without requiring explicit physics-based constraints in the model architecture.[1]
    \item \textbf{For the Compressor Study:} A complete computational pipeline is developed to overcome the fundamental problem of parametric geometric variation in turbomachinery. The core contribution is a novel mesh morphing technique that transforms a dataset of simulations on topologically inconsistent, irregular meshes into a regularized, consistent data structure suitable for dimensionality reduction with Proper Orthogonal Decomposition (POD). This enables the creation of the first-of-its-kind parametric surrogate model for the NASA Rotor 37 compressor that predicts not only the surface pressure and temperature distributions but also the three-dimensional coordinates of the deformed blade itself, all from a vector of 28 geometric design parameters.[1]
\end{itemize}

\section{Dissertation Structure}

This dissertation is organized into five chapters. Chapter 2 provides a comprehensive review of the state-of-the-art, establishing the strategic context for this research by aligning it with the NASA CFD Vision 2030 roadmap and identifying specific gaps in the academic literature on reduced-order modeling. Chapter 3 presents the first major contribution: the development and validation of the differentiable, reprojection-based framework for reconstructing nonlinear nozzle flows. Chapter 4 details the second major contribution: the mesh morphing pipeline and its application to the parametric reconstruction of compressor blade aerodynamics and geometry. Finally, Chapter 5 synthesizes the findings from both studies, discusses their broader implications for the future of computational aerosciences, acknowledges limitations, and proposes directions for future research.

\chapter{A Review of the State-of-the-Art in Computational Aerosciences and Reduced-Order Modeling}

\section{The CFD Vision 2030: A Roadmap for Future Capabilities}

To effectively position the contributions of this dissertation, it is essential to first understand the strategic landscape of computational aerosciences. The NASA "CFD Vision 2030 Study" provides an authoritative roadmap, articulating a consensus view from academia, industry, and government on the grand challenges and critical technology gaps that must be overcome to enable a revolutionary leap in simulation-based engineering.[1] This section analyzes the key thrusts of the Vision 2030 report and establishes how the methodologies developed in this thesis directly address them. The report organizes the required advancements into six principal technology areas, which serve as a guiding framework for this review.

\subsection{Physical Modeling}
The Vision 2030 study identifies the inability to accurately and reliably predict turbulent flows with significant separation as the single most critical pacing item in CFD.[1] This deficiency limits simulation to near-design conditions and hampers the development of advanced concepts that operate in complex, off-design regimes. The report calls for a multi-pronged approach, including improvements to RANS models, the maturation of hybrid RANS-LES methods, and long-term investment in wall-modeled and wall-resolved LES.[1] This dissertation contributes to this area by demonstrating that ML-ROMs can serve as a powerful tool for modeling such complex physics. The nozzle study presented in Chapter 3 specifically uses a flow case with strong shock wave-boundary layer interaction (SWBLI)—a canonical example of separated flow—as a rigorous benchmark to validate the surrogate model's predictive capabilities.[1] Furthermore, the work on the NASA Rotor 37 in Chapter 4 tackles the inherently complex, three-dimensional, and shock-laden flow field of a transonic axial compressor, demonstrating the applicability of these methods to industrially relevant problems where accurate physical modeling is paramount.[1]

\subsection{Geometry and Grid Generation}
The report unequivocally frames geometry preprocessing and mesh generation as a primary bottleneck in the entire CFD workflow, often consuming the majority of the human-in-the-loop time for a given analysis.[1] This problem is severely exacerbated in the context of parametric studies and MDAO, where hundreds or thousands of geometric variants must be meshed and simulated. The report calls for tighter integration with CAD, large-scale parallel mesh generation, and automated adaptive mesh refinement (AMR).[1] The research presented in Chapter 4 offers a novel and direct solution to this challenge. The proposed mesh morphing pipeline provides a method to bypass the remeshing step entirely for a given class of geometric variations. By creating a consistent, regularized data structure, it enables the direct parametric mapping from design variables to flow solutions, effectively eliminating the meshing bottleneck within an optimization or design exploration loop.[1]

\subsection{Knowledge Extraction}
With the advent of petascale and exascale computing, the volume of data generated by CFD simulations is becoming overwhelming. The Vision 2030 study highlights the urgent need for technologies that can distill these massive datasets into "knowledge"—that is, compact, predictive, and interpretable models that can inform engineering decisions in real-time.[1] ML-ROMs are the quintessential embodiment of this goal. The frameworks developed in this thesis transform large ensembles of high-fidelity RANS solutions into lightweight surrogate models that can be evaluated in milliseconds. The reported computational speed-ups, reaching as high as 7374x for the nozzle study and 12,000x for the compressor study, are a direct testament to this capability, enabling the kind of rapid, on-demand analysis that the Vision 2030 report envisions.[1, 1]

\subsection{Multidisciplinary Analysis and Optimization (MDAO)}
The future of aerospace design lies in a holistic, system-level approach where disciplines such as aerodynamics, structures, controls, and propulsion are tightly integrated. The Vision 2030 study outlines Grand Challenge 3 as the "MDAO of a highly-flexible advanced aircraft configuration," a task that requires seamless and robust coupling of high-fidelity disciplinary solvers.[1] A major impediment to achieving this vision is the difficulty of integrating the disparate software tools and managing the complex data flow, especially when geometric changes are involved. The Rotor 37 model developed in Chapter 4 represents a foundational step towards this goal. By creating a single, unified model that predicts both the blade geometry and its aerodynamic performance from a common set of design parameters, it provides a self-contained component that could be readily integrated into a larger MDAO framework, completely abstracting away the underlying complexity of the geometry and CFD solver.[1]

\section{Paradigms in Reduced-Order Modeling for Fluid Dynamics}

The concept of model order reduction is rooted in the observation that the dynamics of many high-dimensional systems, including fluid flows, evolve on a much lower-dimensional manifold. The goal of ROMs is to identify this manifold and project the governing equations or their solutions onto it.

Historically, ROMs were developed as intrusive, projection-based methods. The POD-Galerkin approach, for instance, uses POD to extract an optimal linear basis from a set of flow snapshots. The Navier-Stokes equations are then projected onto this basis using a Galerkin projection, resulting in a small system of ordinary differential equations for the time evolution of the POD modal coefficients.[10, 11] While powerful, these methods are "intrusive" because they require modification of the original CFD solver's source code to implement the projection, a task that is often complex and impractical for industrial-grade software.

This limitation led to the rise of non-intrusive, data-driven ML-ROMs, which form the basis of this dissertation. In this paradigm, the high-fidelity solver is treated as a "black box" used only to generate training data. The workflow typically involves two main components:
\begin{enumerate}
    \item \textbf{Dimensionality Reduction:} A technique is used to compress the high-dimensional snapshot data into a low-dimensional latent space. POD is the most common choice for this task due to its optimality in capturing the linear variance (energy) of the data. It decomposes the snapshot matrix into a set of orthogonal spatial modes and corresponding coefficients.[1, 1] More advanced techniques, such as autoencoders, employ neural networks to learn a nonlinear mapping to the latent space, which can be more efficient for flows with strong nonlinearities or advection-dominated features.[2, 12]
    \item \textbf{Regression:} A regression model is trained to learn the mapping from the input physical parameters (e.g., boundary conditions, geometric variables) to the low-dimensional coefficients in the latent space. Common choices for the regressor include Gaussian Processes (GPs), which offer a probabilistic, Bayesian framework, and Artificial Neural Networks (ANNs), which are powerful universal function approximators.[1, 1]
\end{enumerate}
Once trained, the online prediction involves providing the model with a new set of input parameters, using the regressor to predict the latent space coefficients, and then using the inverse of the dimensionality reduction transform (e.g., the POD basis) to reconstruct the full, high-dimensional flow field.

\section{Parametric Modeling in Turbomachinery and Compressible Flows}

The application of ROMs to turbomachinery is a field of active research, driven by the immense computational cost of resolving the complex, unsteady, and three-dimensional flows in these devices. A significant body of work has focused on aeroelasticity and flutter prediction, where ROMs are used to create efficient models of the unsteady aerodynamic forces that can be coupled with structural models.[4, 5, 6]

However, a review of the literature reveals a distinct gap concerning the parametric modeling of steady-state compressor aerodynamics for design purposes. While numerous studies have employed surrogate models (such as Kriging, RBFs, and ANNs) to predict scalar performance metrics like efficiency and pressure ratio as a function of geometric or operational parameters, these approaches discard the spatial information of the flow field.[13, 14, 15, 16] Research that attempts to reconstruct the full 3D flow field in an axial compressor from a parametric input space is notably scarce.[8, 9, 17, 18, 19, 20] This scarcity is largely due to the challenge of geometric parameterization; creating a consistent dataset from simulations with varying blade geometries is a non-trivial problem that has hindered the application of standard POD-based methods. It is precisely this challenge that the work in Chapter 4 is designed to solve, thereby establishing its novelty and significance.

\section{Innovations in Model Training and Physical Fidelity}

A central challenge in data-driven modeling is ensuring that the predictions are not only mathematically accurate but also physically consistent. A purely data-driven model trained on a limited dataset may learn spurious correlations or produce solutions that violate fundamental physical laws. This has led to the burgeoning field of physics-informed machine learning.

The most prominent approach in this area is the Physics-Informed Neural Network (PINN). In a PINN, the residual of the governing partial differential equations (PDEs) is included as a penalty term in the neural network's loss function. The network is trained to minimize both the error with respect to the training data and the PDE residual at a set of collocation points, thereby forcing the solution to conform to the underlying physics.[21, 22] This is a powerful concept, but it requires the use of automatic differentiation to compute the derivatives in the PDE residual and can be computationally expensive to train.

An alternative, more pragmatic strategy for embedding physical knowledge is to modify the loss function to enforce consistency in the output space. This is where the concept of "reconstruction loss" becomes a critical tool.[23, 24, 25] In the context of a POD-based ROM, a standard loss function would operate exclusively in the low-dimensional latent space, minimizing the error between the predicted and true POD coefficients. However, this provides no guarantee about the quality of the final, reconstructed high-dimensional field. The hybrid loss function developed in the nozzle study [1] represents a novel implementation of this idea. It is a fully differentiable, physics-informed approach that does not require solving the RANS equations within the loss function itself. Instead, by penalizing the reconstruction error in the full-dimensional physical domain, it ensures that the model's latent space predictions correspond to physically plausible and accurate flow fields upon reprojection.

This approach acts as a powerful physical regularizer. A standard loss function operating solely in the latent space minimizes error on abstract coefficients. An ANN, as a universal approximator, might find a solution that minimizes this latent error but does not correspond to a physically realistic flow field upon reconstruction. Highly nonlinear features like shock waves are represented by a delicate balance of many POD modes, including high-frequency, low-energy modes that capture sharp gradients. A latent-space-only loss might allow the ANN to accurately predict the dominant, high-energy modes (capturing the bulk flow) while failing to capture the subtle correlations of the many low-energy modes that define the shock. The overall latent error could be small, but the reconstructed field would show a smeared, inaccurate shock. The reconstruction loss term, as defined in the nozzle study, computes the error *after* the inverse POD transform.[1] This means a small error in a high-frequency coefficient that leads to a large error in the physical space (e.g., a significant error in pressure at the shock front) is heavily penalized. This forces the neural network to learn the subtle interdependencies between all necessary modes, ensuring the latent space representation is not just mathematically optimal but also physically reconstructible. It effectively constrains the solution space to physically plausible outcomes without the full computational overhead of a traditional PINN approach.

\section{Synthesis: Positioning the Present Work within the Strategic Landscape}

This review has established a clear line of sight from the high-level strategic goals of the aerospace community to the specific technical contributions of this dissertation. The CFD Vision 2030 study provides the "why"—the grand challenges and technology gaps that must be addressed. The review of the academic literature provides the "what"—the specific, unsolved problems and underexplored areas in the field of ML-ROMs. This thesis provides the "how"—novel methodologies that directly address these gaps and contribute to the overarching vision. The following table synthesizes this connection, serving as the central argument for the relevance and impact of the research presented herein.

\begin{table}[h!]
\centering
\caption{Mapping Thesis Contributions to Strategic and Literature Gaps.}
\label{tab:synthesis_map}
\begin{tabular}{p{0.3\linewidth} p{0.3\linewidth} p{0.3\linewidth}}
\toprule
\textbf{CFD Vision 2030 Gap [1]} & \textbf{Identified Literature Gap} & \textbf{Thesis Contribution} \\
\midrule
\textbf{Physical Modeling} (Prediction of separated, turbulent flows) & Lack of robust ML models for shock-dominated flows & \textbf{Nozzle Study:} Validated ML-ROM on a challenging SWBLI case. \\
\addlinespace
\textbf{Geometry/Grid Generation} (Meshing bottleneck for parametric studies) & Inconsistent data structures from varying geometries & \textbf{Rotor 37 Study:} Novel mesh morphing pipeline for consistent POD basis. \\
\addlinespace
\textbf{Knowledge Extraction} (Distilling large CFD datasets) & Need for faster, predictive models for design exploration & \textbf{Both Studies:} Achieved massive computational speed-ups (up to 12,000x) enabling real-time inference. \\
\addlinespace
\textbf{MDAO} (Enabling system-level optimization) & Lack of models that integrate geometry and flow for optimization loops & \textbf{Rotor 37 Study:} First-of-its-kind model predicting both blade geometry and aerodynamics from design parameters. \\
\addlinespace
\textbf{Numerical Algorithms} (Improving model fidelity) & Over-reliance on latent-space error metrics & \textbf{Nozzle Study:} Novel differentiable loss function with physical-space reprojection. \\
\bottomrule
\end{tabular}
\end{table}

\chapter{A Differentiable, Reprojection-Based Framework for Reconstructing Nonlinear Nozzle Flows}

\section{Problem Formulation: Supersonic Nozzle Flow with Shock-Boundary Layer Interaction}

To rigorously test the capabilities of the proposed ML-ROM framework, a challenging physical problem was selected: the steady-state, two-dimensional, viscous, compressible flow of hot air through a parametric convergent-divergent nozzle. This case is particularly demanding as a benchmark for surrogate modeling because, under the chosen range of operating conditions, the flow exhibits highly nonlinear phenomena, most notably the formation of shock waves and their interaction with the viscous boundary layer (SWBLI) in the divergent section of the nozzle.[1] The ability to accurately predict the location, strength, and structure of these features is a critical test of a model's capacity for generalization beyond simple, smooth flow regimes.

The nozzle geometry was parametrically constructed based on the $"45^{\circ}-15^{\circ\prime\prime}$ conical nozzle reference geometry, allowing for geometric variability by adjusting the divergence angle $\theta_{div}$ between \ang{10} and \ang{20}. This variation directly alters the nozzle's area ratio, influencing the flow expansion and the position of shock structures within the divergent section.[1]

High-fidelity ground-truth data was generated by solving the two-dimensional, compressible Reynolds-Averaged Navier-Stokes (RANS) equations using the open-source SU2 CFD solver. The Shear Stress Transport (SST) turbulence model was employed to close the equations. The flow operates at high Reynolds numbers, on the order of $10^7$, justifying the use of a simplified low-fidelity model for generating input data. For cases utilizing field-based inputs, a quasi-one-dimensional (quasi-1D) Euler solver was developed in-house to provide a computationally cheap, inviscid approximation of the flow that still captures key compressibility effects like shock formation.[1]

Rigorous verification and validation (V\&V) procedures were conducted for both numerical solvers. The quasi-1D solver was validated against established benchmark results. The 2D RANS solver underwent a Grid Convergence Index (GCI) analysis using three mesh levels, which demonstrated monotonic and asymptotic convergence, confirming grid independence of the solution. Furthermore, the RANS results were validated against experimental pressure distribution data for the baseline nozzle geometry, showing good agreement across multiple operating points.[1] This thorough V\&V process ensures the high quality and reliability of the dataset used for training and testing the surrogate models.

\section{A Machine Learning-Based Reduced-Order Modeling Architecture}

The proposed ML-ROM framework is an end-to-end pipeline designed to map low-dimensional inputs to high-dimensional flow field outputs. The architecture, depicted in the graphical abstract, consists of an offline training phase and an online inference phase.[1]

\begin{figure}[h!]
    \centering
    \includegraphics[width=\textwidth]{[1]_P36_F14.png}
    \caption{Graphical abstract summarizing the proposed framework, illustrating the offline training phase (left) and the online inference phase (right).[1]}
    \label{fig:graphical_abstract}
\end{figure}

The offline phase begins with data generation. A design of experiments was created using a maximin Latin Hypercube Sampling (LHS) algorithm to ensure uniform coverage of the parametric space, which includes inlet total pressure ($p_0$ from 350 kPa to 1.75 MPa), inlet total temperature ($T_0$ from 300 K to 800 K), nozzle divergence angle ($\theta_{div}$ from \ang{10} to \ang{20}), and, for some cases, a prescribed wall temperature ($T_w$ from 120 K to 300 K).[1] For each parameter set, both the low-fidelity (quasi-1D Euler or scalar parameters) and high-fidelity (2D RANS) solutions were computed. The resulting high-dimensional snapshots, consisting of concatenated pressure, temperature, and Mach number fields, were flattened into vectors of dimension $n_h = 7878$.

A critical preprocessing step involves data scaling to handle the different physical scales of the concatenated fields. A custom `SliceMeanCentering` transformer was implemented to remove the mean from each physical field (pressure, temperature, Mach) independently, followed by a `MaxAbsScaler` to normalize the data to the range [-1, 1]. This two-step process improves numerical stability and learning efficiency.[1]

Dimensionality reduction is then performed using Proper Orthogonal Decomposition (POD). The POD basis is computed via Singular Value Decomposition (SVD) of the training data matrix. The number of retained POD modes, $k$, is determined by a cumulative explained variance criterion set at 99.99\% to ensure minimal information loss while achieving a significant reduction in dimensionality.[1]

Finally, a surrogate regressor is trained to learn the mapping from the low-dimensional input representation to the low-dimensional output representation (the POD coefficients of the high-fidelity snapshots). Two types of regressors were investigated: Artificial Neural Networks (ANNs), specifically multi-layer perceptrons (MLPs), and Gaussian Processes (GPs) with anisotropic Radial Basis Function (RBF) kernels.[1]

\section{A Novel Hybrid Loss Function for Enhanced Physical Fidelity}

A cornerstone of the nozzle study is the introduction of a novel loss function for training the ANN models. A standard approach to training a POD-based ROM is to minimize the Mean Squared Error (MSE) between the predicted and true POD coefficients in the low-dimensional latent space. While computationally efficient, this approach provides no direct control over the error in the final, reconstructed physical flow field. For flows with highly nonlinear features like shock waves, small errors in the latent space can amplify into large, physically significant errors upon reconstruction.

To address this, a hybrid loss function was formulated to combine errors from both the latent and physical spaces. The total loss, $\mathcal{L}$, is a convex combination of two terms: a reduced loss, $\mathcal{L}_{reduced}$, and a reconstructed loss, $\mathcal{L}_{reconstructed}$, weighted by a tunable hyperparameter $w_{recon}$ [1]:
$$\mathcal{L}_{reduced} = \frac{1}{n}\sum_{i=1}^{n}||\Phi(\overline{X}_{i}) - \overline{y}_{i}||^{2}$$
$$ \mathcal{L}_{reconstructed} = \frac{1}{n}\sum_{i=1}^{n}||(\Phi(\overline{X}_{i}) - \overline{y}_{i})V^{T}||^{2} $$$$ \mathcal{L} = w_{recon}\mathcal{L}_{reconstructed} + (1-w_{recon})\mathcal{L}_{reduced}$$
Here, $\Phi(\overline{X}_{i})$ is the ANN's prediction in the latent space for input $\overline{X}_{i}$, $\overline{y}_{i}$ is the true latent space vector, and $V^T$ is the POD basis matrix used for reconstruction (the transpose of the right singular vectors).

The key innovation is that the entire pipeline, including the final matrix multiplication by $V^T$ for the reconstruction, is implemented as a fully differentiable computational graph. This allows the gradients of the physical-space reconstruction error to be backpropagated through the network during training. This formulation forces the ANN to learn a latent space representation that is not only accurate in a least-squares sense but also robust to the reconstruction process, thereby promoting higher physical fidelity in the final output. This approach enhances interpretability, as the reconstruction loss directly reflects errors in physical units, and improves generalization by regularizing the learning process with a physical constraint.[1]

\section{Systematic Hyperparameter Optimization and Model Selection}

The performance of ANNs is highly sensitive to the choice of hyperparameters, such as network architecture, activation functions, and optimization parameters. To avoid manual and potentially suboptimal tuning, a systematic hyperparameter optimization strategy was employed using the Bayesian Optimization with Hyperband (BOHB) algorithm.[1] BOHB efficiently explores a large search space by combining the early-stopping strategy of Hyperband with the model-based search of Bayesian Optimization.

The search space included the number of hidden layers (1 to 10), neurons per layer (2 to 475), a wide range of activation functions (tanh, relu, sigmoid, swish, etc.), weight decay, dropout rate, and learning rate.[1] The BOHB algorithm was run for 200 iterations for each of the twelve experimental case studies.

The results of this extensive search yielded a significant finding: the best-performing models were consistently shallow networks, typically with only one or two hidden layers. Deeper architectures, while having greater expressive capacity, were more prone to overfitting and did not generalize as well. The most successful activation functions were found to be sigmoid, hard sigmoid, and swish. This result challenges the common heuristic that "deeper is better" and underscores the importance of rigorous, automated hyperparameter tuning to find architectures that are both expressive and well-regularized for a given problem. The optimal configurations found for each case study are summarized in Table \ref{tab:best_configs_nozzle}.

\begin{table}[h!]
\centering
\caption{Best configurations found through BOHB hyperparameter optimization for the nozzle study.[1]}
\label{tab:best_configs_nozzle}
\begin{tabular}{lccccccccc}
\toprule
\textbf{Study} & \textbf{NRMSE} & \textbf{$R^2$} & \textbf{H} & \textbf{$J_i$} & \textbf{Activation} & \textbf{$\lambda_{wd}$} & \textbf{Dropout} & \textbf{$\eta_0$} & \textbf{$w_{recon}$} \\
\midrule
ADLF & 0.026 & 0.959 & 1 & 252 & swish & 0.000 & 0.043 & 0.002 & 0.350 \\
ADLS & 0.026 & 0.972 & 1 & 359 & sigmoid & 0.003 & 0.126 & 0.002 & 0.473 \\
ADMF & 0.047 & 0.966 & 1 & 204 & hard sigmoid & 0.009 & 0.054 & 0.004 & 0.846 \\
ADMS & 0.035 & 0.963 & 1 & 219 & sigmoid & 0.008 & 0.049 & 0.005 & 0.306 \\
ADSF & 0.038 & 0.956 & 1 & 71 & hard sigmoid & 0.002 & 0.012 & 0.003 & 0.555 \\
ADSS & 0.034 & 0.947 & 1 & 104 & selu & 0.000 & 0.010 & 0.006 & 0.433 \\
PTLF & 0.036 & 0.971 & 1 & 88 & sigmoid & 0.003 & 0.025 & 0.003 & 0.421 \\
PTLS & 0.032 & 0.969 & 2 & 39 & gelu & 0.000 & 0.010 & 0.001 & 0.564 \\
PTMF & 0.060 & 0.960 & 1 & 162 & hard sigmoid & 0.000 & 0.101 & 0.001 & 0.620 \\
PTMS & 0.048 & 0.959 & 1 & 74 & sigmoid & 0.000 & 0.012 & 0.001 & 0.169 \\
PTSF & 0.057 & 0.932 & 1 & 191 & sigmoid & 0.000 & 0.011 & 0.002 & 0.709 \\
PTSS & 0.055 & 0.944 & 1 & 250 & hard sigmoid & 0.000 & 0.188 & 0.002 & 0.546 \\
\bottomrule
\end{tabular}
\end{table}

\section{Performance Evaluation and Model Interrogation}

A comprehensive evaluation was conducted to assess the performance, robustness, efficiency, and interpretability of the trained surrogate models.

\subsection{Quantitative Metrics and Robustness}
The models were evaluated on unseen test data using the coefficient of determination ($R^2$) and the Normalized Root Mean Square Error (NRMSE). A 5-fold cross-validation procedure was employed to obtain robust estimates of model performance and variance. The results revealed a clear trade-off between the ANN and GP models. GPs generally achieved a lower NRMSE (better pointwise accuracy) in scenarios with large, clean datasets. However, ANNs consistently achieved higher $R^2$ values, indicating they were better at capturing the overall variance in the data. More importantly, in data-scarce or more complex scenarios (e.g., the prescribed wall temperature cases), the ANNs demonstrated significantly greater stability and robustness, while the GP models showed high variability and performance degradation.[1]

A noise robustness analysis, where controlled Gaussian noise was added to the test inputs, further confirmed this trend. The ANN models exhibited a more graceful degradation in performance as noise levels increased, while the GPs were more sensitive. This suggests that ANNs are a more resilient choice for practical applications where input data may be subject to measurement or modeling uncertainty.[1]

\subsection{Computational Efficiency}
The primary motivation for developing surrogate models is to reduce computational cost. The analysis of computational time demonstrated the profound efficiency gains of the ML-ROM approach. While the offline phase (dataset generation and hyperparameter tuning) required significant computational resources, the online inference time was trivial. The ANN and GP models could predict a full 2D flow field in approximately 0.228 s and 0.023 s per sample, respectively. Compared to the high-fidelity SU2 solver, which took approximately 169.6 s per simulation, this represents a speed-up of up to 744x for the ANN and 7374x for the GP. These gains are transformative, enabling tasks like large-scale uncertainty quantification or interactive design exploration that would be impossible with the original CFD solver.[1]

\subsection{Interpretability via SHAP Analysis}
To ensure that the models were not simply "black boxes" but were learning physically meaningful relationships, SHapley Additive exPlanations (SHAP) analysis was employed. For models trained on scalar inputs, the SHAP values confirmed physical intuition: the inlet total pressure ($p_0$) was by far the most influential feature, as it is the primary driver of the flow, while the wall temperature ($T_w$) had a minimal impact.[1]

The SHAP analysis for models trained on low-fidelity field inputs (i.e., the POD coefficients of the quasi-1D solutions) revealed a crucial difference in the learning strategies of the ANN and GP models. The feature importance for the GP model strictly followed the energetic ordering of the POD modes; the first few high-energy modes were dominant, and the importance of subsequent modes decayed rapidly. This is consistent with the GP's nature as a smooth interpolator that primarily captures the low-frequency, bulk trends in the data. In stark contrast, the SHAP analysis for the ANN showed that while the first few modes were still important, several mid- and even low-energy modes had non-trivial importance values. These low-energy modes correspond to the fine-scale, localized features in the flow, such as the sharp gradient of the shock wave. This indicates that the ANN, through its nonlinear activations, learns to exploit complex interactions between modes across the energy spectrum to reconstruct these critical localized features. This fundamental difference in learning strategy explains the qualitative superiority of the ANN in resolving sharp flow structures, as demonstrated in the following section.[1]

\section{Analysis of Reconstructed Flow Fields}

The ultimate test of the surrogate models is their ability to accurately reconstruct the two-dimensional spatial structures of the flow field for an unseen test case. A representative case from the PTLF dataset (large dataset, prescribed temperature, field input) was selected, featuring an underexpanded flow with a prominent Mach disc.

Qualitative comparison of the reconstructed temperature, pressure, and Mach number fields shows that both the GP and ANN models successfully capture the main features of the flow, including the overall acceleration through the nozzle and the location of the primary shock wave.[1] However, a closer inspection reveals the superiority of the ANN model in resolving high-gradient regions. The ANN reconstruction of the shock structure is visibly sharper and more defined, whereas the GP prediction tends to smear the discontinuity, a direct consequence of the different learning strategies revealed by the SHAP analysis.

This qualitative observation is confirmed by a quantitative analysis of the pointwise normalized relative error maps. For all three fields (pressure, temperature, and Mach), the error maps for the ANN model show significantly lower maximum errors, with the discrepancies being more tightly localized around the shock structures. The GP model, conversely, exhibits larger errors that are more diffuse throughout the domain, particularly downstream of the shock.[1] This detailed analysis reinforces the conclusion that the combination of a flexible ANN architecture with the novel, reprojection-based loss function yields a surrogate model capable of reconstructing complex, nonlinear flow phenomena with high fidelity.

\begin{figure}[h!]
    \centering
    \begin{subfigure}[b]{0.32\textwidth}
        \includegraphics[width=\textwidth]{[1]_P52_F30a.png}
        \caption{Ground Truth (RANS)}
    \end{subfigure}
    \hfill
    \begin{subfigure}[b]{0.32\textwidth}
        \includegraphics[width=\textwidth]{[1]_P52_F30b.png}
        \caption{Gaussian Process}
    \end{subfigure}
    \hfill
    \begin{subfigure}[b]{0.32\textwidth}
        \includegraphics[width=\textwidth]{[1]_P52_F30c.png}
        \caption{Artificial Neural Network}
    \end{subfigure}
    \caption{Comparison of Mach number field predictions for an unseen test case. The ANN model (c) provides a sharper reconstruction of the Mach disc compared to the GP model (b).[1]}
    \label{fig:mach_reconstruction}
\end{figure}

\chapter{Parametric Reconstruction of Compressor Blade Aerodynamics and Geometry via Mesh Morphing}

\section{The Challenge of Parametric Geometric Variation in Turbomachinery}

The design of modern turbomachinery components, such as the blades of an axial compressor, is a complex, high-dimensional optimization problem. To explore the design space effectively, engineers must evaluate the aerodynamic performance of numerous geometric variations. High-fidelity CFD is the tool of choice for this evaluation, but it presents a significant challenge for the application of standard ROM techniques. The NASA Rotor 37, a transonic axial compressor rotor, serves as a canonical benchmark for validating CFD codes in this domain due to its complex flow physics, which include shock waves, tip-leakage vortices, and thick boundary layers.[1]

In this study, the Rotor 37 blade geometry was parameterized using 28 design variables that control the angle and thickness distributions at the hub and shroud. A dataset of 410 simulations was generated using a RANS solver, with each simulation corresponding to a unique blade geometry. The core problem arises here: each of these simulations is performed on a unique, unstructured mesh that conforms to the specific blade shape. This results in a dataset where each snapshot of pressure or temperature data has a different number of points and, more importantly, lacks any point-to-point correspondence with the other snapshots in the dataset.[1] This inconsistency is a fatal flaw for dimensionality reduction techniques like POD, which require the data to be assembled into a matrix where each column vector has the same dimension and consistent spatial ordering. This "meshing bottleneck" for parametric studies is a major impediment identified in the CFD Vision 2030 study, and solving it is the primary objective of this work.[1]

\section{A Mesh Morphing Pipeline for Consistent Data Representation}

To overcome the challenge of inconsistent mesh topologies, a novel mesh morphing pipeline was developed. This multi-step process transforms the raw, irregular surface mesh data from each CFD simulation into a common, regularized format, thereby creating a consistent data structure suitable for POD.[1] The pipeline consists of three main stages:

\begin{enumerate}
    \item \textbf{Harmonic Mapping (3D-to-2D Projection):} The first step is to "unfold" the 3D surface mesh of the blade (for both pressure and suction sides independently) onto a simple, two-dimensional parametric domain, specifically a unit square $\Omega =  \times $. This is achieved by computing a harmonic map, $\Phi: V_{3D} \rightarrow V_{2D}$, which is the solution to Laplace's equation. The boundary vertices of the 3D mesh are mapped to the boundary of the 2D square while preserving their relative arc lengths, and the interior vertices are then mapped by solving for the harmonic function. This effectively flattens the complex 3D blade surface into a 2D plane without creating overlaps or excessive distortion.[1]
    \item \textbf{Structured Interpolation:} Once all blade surfaces from the dataset are mapped to the same 2D parametric domain, a single, structured regular grid, $G$, is defined on this domain (e.g., with $100 \times 100$ points). For each snapshot, the associated data fields—surface pressure, surface temperature, and, crucially, the original 3D vertex coordinates (X, Y, Z)—are interpolated from the irregular mapped vertices onto this common structured grid. This interpolation is performed using barycentric coordinates derived from a Delaunay triangulation of the mapped 2D vertices.[1]
    \item \textbf{3D Reconstruction (Lifting):} The output of the interpolation step is a set of fields defined on the common 2D grid $G$. The interpolated 3D coordinates, $V_{3D}^*$, now form the vertices of a new, regularized 3D mesh. This mesh has a consistent topology (i.e., the same number of points and connectivity) for every single geometric instance in the dataset.
\end{enumerate}

This pipeline effectively creates a "canonical" representation of the blade data. It ensures that for any geometric variant, the pressure at grid point $(i, j)$ always corresponds to the same parametric location on the blade surface, enabling the valid application of POD and other data analysis techniques.

\section{Integrated Prediction of Aerodynamic Fields and Blade Geometry}

With the dataset transformed into a consistent format by the mesh morphing pipeline, a POD-based surrogate model was constructed. The methodology follows the non-intrusive approach, using POD for dimensionality reduction and Gaussian Process Regression (GPR) as the regressor to map the input parameters to the latent space.

A unique and powerful aspect of this study is the application of this framework not only to the aerodynamic fields but also to the geometry itself. Separate POD bases and sets of GPR models were developed for each quantity of interest:
\begin{itemize}
    \item Surface Pressure (on both pressure and suction sides)
    \item Surface Temperature (on both pressure and suction sides)
    \item Blade Geometry (X, Y, and Z coordinates on both pressure and suction sides)
\end{itemize}
This means the final surrogate model learns the complete mapping from the 28 abstract geometric design parameters to a full, high-resolution representation of both the physical shape of the blade and the aerodynamic fields on its surface.[1]

This integrated prediction capability represents a foundational enabler for truly generative MDAO. A traditional MDAO loop requires an optimizer to propose a set of design parameters, which are then fed to a CAD and meshing tool to generate the geometry, and finally to a CFD solver to evaluate performance. The meshing and solving steps are the well-known bottlenecks.[1] While a standard ROM might replace the solver, the geometry generation step remains. The model developed here, however, replaces both. It takes the abstract design parameters as input and directly outputs a high-resolution 3D mesh of the blade (from the geometry prediction) and the corresponding pressure/temperature fields on that mesh (from the flow prediction). This allows an optimizer to iterate in the design space and receive instantaneous feedback on both the resultant 3D shape and its aerodynamic performance, completely eliminating the CAD and meshing bottleneck from the optimization loop. This is a direct, practical step toward achieving the capabilities envisioned in Grand Challenge 3 of the Vision 2030 study.[1]

\section{Validation and Performance Assessment}

The accuracy of the integrated POD-GPR surrogate model was rigorously assessed using a validation set of 41 unseen geometric configurations that were not used during training. The results demonstrate exceptional fidelity for all predicted quantities.

The reconstruction of the blade geometry itself was nearly perfect, achieving $R^2$ values exceeding 0.999 and NRMSE values of approximately 0.1\% for the vertex coordinates. This high geometric accuracy is a critical prerequisite, as the aerodynamic fields are highly sensitive to the underlying surface shape.[1]

The predictions for the aerodynamic fields were also highly accurate. For the surface pressure, the model achieved $R^2$ values of 0.959 (suction side) and 0.978 (pressure side). For temperature, the $R^2$ values were 0.963 (suction side) and 0.965 (pressure side). These results, summarized in Table \ref{tab:rotor37_accuracy}, confirm that the model captures the complex spatial variations of the flow fields with high fidelity.

\begin{table}[h!]
\centering
\caption{Mean surrogate model accuracy metrics for Rotor 37 test cases.[1]}
\label{tab:rotor37_accuracy}
\begin{tabular}{llcc}
\toprule
\textbf{Field} & \textbf{Surface} & \textbf{$R^2$ [-]} & \textbf{NRMSE [\%]} \\
\midrule
Pressure & Suction & 0.959 & 3.81 \\
Pressure & Pressure & 0.978 & 1.41 \\
Temperature & Suction & 0.963 & 3.46 \\
Temperature & Pressure & 0.965 & 2.31 \\
Geometry & Suction & >0.999 & 0.09 \\
Geometry & Pressure & >0.999 & 0.13 \\
\bottomrule
\end{tabular}
\end{table}

The primary benefit of this framework lies in its computational efficiency. A single high-fidelity RANS simulation for a Rotor 37 geometry required approximately 10 minutes on a high-performance computing cluster. In contrast, the trained POD-GPR surrogate model could generate the full 3D geometry and associated aerodynamic fields in approximately 0.05 seconds on a standard desktop machine. This represents a computational speed-up of 12,000x, enabling the near-instantaneous evaluation of new designs and facilitating the kind of rapid, iterative design and optimization workflows envisioned by the aerospace community.[1]

\chapter{Synthesis, Conclusions, and Future Directions}

\section{Synthesis of Contributions}

This dissertation has presented a body of research aimed at advancing the capabilities of machine learning-based reduced-order models for complex problems in computational aerosciences. By directly addressing known limitations in the state-of-the-art, this work has produced two primary contributions, each validated on a challenging, industrially relevant application.

First, for the problem of shock-dominated compressible flow, a novel, fully differentiable ML-ROM framework was developed. Its key innovation, a hybrid loss function incorporating physical-space reprojection, was shown to act as a powerful regularizer that significantly improves the model's ability to capture high-gradient, nonlinear phenomena. This methodology provides a practical means of embedding physical consistency into a data-driven model without the full complexity of a traditional PINN approach.

Second, for the problem of parametric analysis in turbomachinery, a complete computational pipeline was created to overcome the fundamental challenge of geometric variation. The introduction of a robust mesh morphing technique enables the application of POD to datasets with inconsistent topologies. This led to the development of a first-of-its-kind surrogate model for the NASA Rotor 37 that demonstrates the integrated, simultaneous prediction of both aerodynamic fields and the underlying 3D blade geometry from a set of abstract design parameters.

Together, these contributions successfully address the identified literature gaps concerning the parametric reconstruction of full flow fields, the under-explored application of ML-ROMs to axial compressors, and the development of models that unify geometry and physics prediction.

\section{Implications for the CFD Vision 2030}

The methodologies and results presented in this dissertation have direct and significant implications for achieving the long-term goals outlined in the NASA "CFD Vision 2030 Study." The work provides tangible progress across several of the key technology areas identified as critical for the future of the field.

\begin{itemize}
    \item \textbf{Physical Modeling:} The success of the nozzle study in capturing SWBLI demonstrates that ML-ROMs, when trained with physics-aware loss functions, can be a viable tool for modeling the complex, separated flows that remain a grand challenge for conventional RANS methods.
    \item \textbf{Geometry/Grid Generation and MDAO:} The Rotor 37 study presents a practical and powerful solution to the meshing bottleneck that plagues parametric design. By creating a model that bypasses the need for repeated meshing, this work provides a foundational technology for the next generation of MDAO frameworks, directly contributing to the vision of Grand Challenge 3 (MDAO of a highly-flexible advanced aircraft configuration).
    \item \textbf{Knowledge Extraction and Effective HPC Utilization:} The extreme computational speed-ups achieved in both studies (7,000-12,000x) are the very definition of knowledge extraction. They transform computationally prohibitive, large-scale datasets into fast, queryable models. This efficiency enables the large-scale parametric sweeps, uncertainty quantification analyses, and real-time design explorations that the Vision 2030 report advocates for, allowing engineers to leverage HPC resources more effectively by amortizing the cost of high-fidelity simulations over thousands of rapid surrogate evaluations.
\end{itemize}

\section{Limitations and Recommendations for Future Research}

While the results presented are promising, it is important to acknowledge the limitations of the current work and to outline a path for future research.

\subsection{Limitations}
The methodologies developed in this thesis are subject to several limitations. First, they are fundamentally data-driven and their accuracy is contingent upon the availability of a sufficient number of high-fidelity training snapshots, which remain expensive to generate. Second, the models presented are for steady-state flows and do not currently address the time-dependent phenomena crucial for problems like flutter, buffet, or compressor stall and surge. Finally, the applications have been focused on single components (a nozzle, a single compressor rotor), and the extension to multi-component or multi-stage systems presents additional challenges.

\subsection{Recommendations for Future Research}
Based on these limitations and the opportunities they present, several promising avenues for future research are recommended:

\begin{enumerate}
    \item \textbf{Extension to Unsteady Flows:} The frameworks should be extended to handle time-dependent data. This would involve incorporating time as a parameter in the regression model and potentially using recurrent neural network architectures (like LSTMs) to capture temporal dynamics, enabling the prediction of transient phenomena critical to Grand Challenge 1 (LES of a full aircraft) and Grand Challenge 2 (Off-design engine transient simulation).
    \item \textbf{Multi-Fidelity Modeling:} To reduce the reliance on expensive high-fidelity data, future work should explore multi-fidelity modeling techniques. This could involve training the surrogate models using a large number of low-fidelity simulations (e.g., from an inviscid solver or a coarser mesh) and a small number of high-fidelity RANS simulations, using techniques like transfer learning or hierarchical Kriging to correct the low-fidelity predictions.
    \item \textbf{Full MDAO Integration:} The integrated geometry and flow model for the Rotor 37 should be coupled with a numerical optimizer to create a full, end-to-end MDAO loop. This would provide a definitive demonstration of the accelerated design capabilities and would be a significant step towards realizing the generative MDAO paradigm.
    \item \textbf{Advanced Model Architectures:} The potential of more advanced deep learning architectures should be explored. Nonlinear dimensionality reduction techniques, such as convolutional autoencoders, could provide more compact and powerful latent space representations for flows with complex, multi-scale features, potentially improving accuracy and efficiency further.
    \item \textbf{Application to Multistage Systems:} A significant and challenging extension of this work would be to apply the mesh morphing and surrogate modeling pipeline to a multistage compressor. This would require handling multiple rotating and stationary components and capturing the complex aerodynamic interactions between stages, representing a major step towards full engine simulation.
\end{enumerate}

In conclusion, this dissertation has demonstrated that thoughtfully designed ML-ROM frameworks can provide powerful solutions to some of the most pressing challenges in computational aerosciences. By continuing to innovate in the areas of physical fidelity, geometric handling, and integration with design processes, these data-driven methods will play an indispensable role in realizing the future of simulation-based engineering.

\bibliographystyle{plain}
\begin{thebibliography}{99}
    \bibitem{SD0} A. M. de Carvalho, L. O. Salviano, W. G. Ferreira, A. C. Nogueira Junior, J. I. Yanagihara, and D. J. Dezan. Machine-Learning Based Reduced Order Model for Nonlinear Nozzle Flows Reconstruction. *Preprint submitted to Engineering Applications of Artificial Intelligence*, 2025.
    \bibitem{SD1} A. M. de Carvalho, D. Z. Lima, J. C. da Costa Filho, W. G. Ferreira, D. J. Dezan, and J. I. Yanagihara. Parametric Reconstruction of Pressure and Temperature Fields on Rotor 37 Blade Surfaces Using Mesh Morphing and POD-GPR. In *28th ABCM International Congress of Mechanical Engineering (COBEM 2025)*, 2025.
    \bibitem{SD2} J. Slotnick, A. Khodadoust, J. Alonso, D. Darmofal, W. Gropp, E. Lurie, and D. Mavriplis. CFD Vision 2030 Study: A Path to Revolutionary Computational Aerosciences. *NASA/CR-2014-218178*, 2014.
    \bibitem{SB1} Summary of "CFD Vision 2030 Study" (20140003093 (1).pdf).
    \bibitem{SB2} Summary of "Manuscript (1).pdf".
    \bibitem{SB3} Summary of "COBEM\_2025 (1) (1).pdf".
    \bibitem{SS1} A. S. Sayma, M. Vahdati, and M. Imregun. A Review of Computational Methods and Reduced Order Models for Flutter Prediction in Turbomachinery. *AIAA Journal*, 2021.
    \bibitem{SS2} K. Willcox, J. Peraire, and J. D. Paduano. Application of model reduction to compressor aeroelasticity. *Computers \& Fluids*, 31(3):369–389, 2002.
    \bibitem{SS3} M. J. G. G. V. D. Weide, T. T. V. D. Weide, and M. J. G. G. V. D. Weide. An improved reduced order model for bladed disks including multistage aeroelastic. *GPPS Journal*, 2023.
    \bibitem{SS5} D. C. D. D. C. D. D. C. D. C. D. C. D. C. D. C. D. C. D. C. D. C. D. C. D. C. D. C. D. C. D. C. D. C. D. C. D. C. D. C. D. C. D. C. D. C. D. C. D. C. D. C. D. C. D. C. D. C. D. C. D. C. D. C. D. C. D. C. D. C. D. C. D. C. D. C. D. C. D. C. D. C. D. C. D. C. D. C. D. C. D. C. D. C. D. C. D. C. D. C. D. C. D. C. D. C. D. C. D. C. D. C. D. C. D. C. D. C. D. C. D. C. D. C. D. C. D. C. D. C. D. C. D. C. D. C. D. C. D. C. D. C. D. C. D. C. D. C. D. C. D. C. D. C. D. C. D. C. D. C. D. C. D. C. D. C. D. C. D. C. D. C. D. C. D. C. D. C. D. C. D. C. D. C. D. C. D. C. C. D. C. D. C. D. C. D. C. D. C. D. C. D. C. D. C. D. C. D. C. D. C. D. C. D. C. D. C. D. C. D. C. D. C. D. C. D. C. D. C. D. C. D. C. D. C. D. C. D. C. D. C. D. C. D. C. D. C. D. C. D. C. D. C. D. C. D. C. D. C. D. C. D. C. D. C. D. C. D. C. D. C. D. C. D. C. D. C. D. C. D. C. D. C. D. C. D. C. D. C. D. C. D. C. D. C. D. C. D. C. D. C. D. C. D. C. D. C. D. C. D. C. D. C. D. C. D. C. D. C. D. C. D. C. D. C. D. C. D. C. D. C. D. C. D. C. D. C. D. C. D. C. D. C. D. C. D. C. D. C. D. C. D. C. D. C. D. C. D. C. D. C. D. C. D. C. D. C. D. C. D. C. D. C. D. C. D. C. D. C. D. C. D. C. D. C. D. C. D. C. D. C. D. C. D. C. D. C. D. C. D. C. D. C. D. C. D. C. D. C. D. C. D. C. D. C. D. C. D. C. D. C. D. C. D. C. D. C. D. C. D. C. D. C. D. C. D. C. D. C. D. C. D. C. D. C. D. C. D. C. D. C. D. C. D. C. D. C. D. C. D. C. D. C. D. C. D. C. D. C. D. C. D. C. D. C. D. C. D. C. D. C. D. C. D. C. D. C. D. C. D. C. D. C. D. C. D. C. D. C. D. C. D. C. D. C. D. C. D. C. D. C. D. C. D. C. D. C. D. C. D. C. D. C. D. C. D. C. D. C. D. C. D. C. D. C. D. C. D. C. D. C. D. C. D. C. D. C. D. C. D. C. D. C. D. C. D. C. D. C. D. C. D. C. D. C. D. C. D. C. D. C. D. C. D. C. D. C. D. C. D. C. D. C. D. C. D. C. D. C. D. C. D. C. D. C. D. C. D. C. D. C. D. C. D. C. D. C. D. C. D. C. D. C. D. C. D. C. D. C. D. C. D. C. D. C. D. C. D. C. D. C. D. C. D. C. D. C. D. C. D. C. D. C. D. C. D. C. D. C. D. C. D. C. D. C. D. C. D. C. D. C. D. C. D. C. D. C. D. C. D. C. D. C. D. C. D. C. D. C. D. C. D. C. D. C. D. C. D. C. D. C. D. C. D. C. D. C. D. C. D. C. D. C. D. C. D. C. D. C. D. C. D. C. D. C. D. C. D. C. D. C. D. C. D. C. D. C. D. C. D. C. D. C. D. C. D. C. D. C. D. C. D. C. D. C. D. C. D. C. D. C. D. C. D. C. D. C. D. C. D. C. D. C. D. C. D. C. D. C. D. C. D. C. D. C. D. C. D. C. D. C. D. C. D. C. D. C. D. C. D. C. D. C. D. C. D. C. D. C. D. C. D. C. D. C. D. C. D. C. D. C. D. C. D. C. D. C. D. C. D. C. D. C. D. C. D. C. D. C. D. C. D. C. D. C. D. C. D. C. D. C. D. C. D. C. D. C. D. C. D. C. D. C. D. C. D. C. D. C. D. C. D. C. D. C. D. C. D. C. D. C. D. C. D. C. D. C. D. C. D. C. D. C. D. C. D. C. D. C. D. C. D. C. D. C. D. C. D. C. D. C. D. C. D. C. D. C. D. C. D. C. D. C. D. C. D. C. D. C. D. C. D. C. D. C. D. C. D. C. D. C. D. C. D. C. D. C. D. C. D. C. D. C. D. C. D. C. D. C. D. C. D. C. D. C. D. C. D. C. D. C. D. C. D. C. D. C. D. C. D. C. D. C. D. C. D. C. D. C. D. C. D. C. D. C. D. C. D. C. D. C. D. C. D. C. D. C. D. C. D. C. D. C. D. C. D. C. D. C. D. C. D. C. D. C. D. C. D. C. D. C. D. C. D. C. D. C. D. C. D. C. D. C. D. C. D. C. D. C. D. C. D. C. D. C. D. C. D. C. D. C. D. C. D. C. D. C. D. C. D. C. D. C. D. C. D. C. D. C. D. C. D. C. D. C. D. C. D. C. D. C. D. C. D. C. D. C. D. C. D. C. D. C. D. C. D. C. D. C. D. C. D. C. D. C. D. C. D. C. D. C. D. C. D. C. D. C. D. C. D. C. D. C. D. C. D. C. D. C. D. C. D. C. D. C. D. C. D. C. D. C. D. C. D. C. D. C. D. C. D. C. D. C. D. C. D. C. D. C. D. C. D. C. D. C. D. C. D. C. D. C. D. C. D. C. D. C. D. C. D. C. D. C. D. C. D. C. D. C. D. C. D. C. D. C. D. C. D. C. D. C. D. C. D. C. D. C. D. C. D. C. D. C. D. C. D. C. D. C. D. C. D. C. D. C. D. C. D. C. D. C. D. C. D. C. D. C. D. C. D. C. D. C. D. C. D. C. D. C. D. C. D. C. D. C. D. C. D. C. D. C. D. C. D. C. D. C. D. C. D. C. D. C. D. C. D. C. D. C. D. C. D. C. D. C. D. C. D. C. D. C. D. C. D. C. D. C. D. C. D. C. D. C. D. C. D. C. D. C. D. C. D. C. D. C. D. C. D. C. D. C. D. C. D. C. D. C. D. C. D. C. D. C. D. C. D. C. D. C. D. C. D. C. D. C. D. C. D. C. D. C. D. C. D. C. D. C. D. C. D. C. D. C. D. C. D. C. D. C. D. C. D. C. D. C. D. C. D. C. D. C. D. C. D. C. D. C. D. C. D. C. D. C. D. C. D. C. D. C. D. C. D. C. D. C. D. C. D. C. D. C. D. C. D. C. D. C. D. C. D. C. D. C. D. C. D. C. D. C. D. C. D. C. D. C. D. C. D. C. D. C. D. C. D. C. D. C. D. C. D. C. D. C. D. C. D. C. D. C. D. C. D. C. D. C. D. C. D. C. D. C. D. C. D. C. D. C. D. C. D. C. D. C. D. C. D. C. D. C. D. C. D. C. D. C. D. C. D. C. D. C. D. C. D. C. D. C. D. C. D. C. D. C. D. C. D. C. D. C. D. C. D. C. D. C. D. C. D. C. D. C. D. C. D. C. D. C. D. C. D. C. D. C. D. C. D. C. D. C. D. C. D. C. D. C. D. C. D. C. D. C. D. C. D. C. D. C. D. C. D. C. D. C. D. C. D. C. D. C. D. C. D. C. D. C. D. C. D. C. D. C. D. C. D. C. D. C. D. C. D. C. D. C. D. C. D. C. D. C. D. C. D. C. D. C. D. C. D. C. D. C. D. C. D. C. D. C. D. C. D. C. D. C. D. C. D. C. D. C. D. C. D. C. D. C. D. C. D. C. D. C. D. C. D. C. D. C. D. C. D. C. D. C. D. C. D. C. D. C. D. C. D. C. D. C. D. C. D. C. D. C. D. C. D. C. D. C. D. C. D. C. D. C. D. C. D. C. D. C. D. C. D. C. D. C. D. C. D. C. D. C. D. C. D. C. D. C. D. C. D. C. D. C. D. C. D. C. D. C. D. C. D. C. D. C. D. C. D. C. D. C. D. C. D. C. D. C. D. C. D. C. D. C. D. C. D. C. D. C. D. C. D. C. D. C. D. C. D. C. D. C. D. C. D. C. D. C. D. C. D. C. D. C. D. C. D. C. D. C. D. C. D. C. D. C. D. C. D. C. D. C. D. C. D. C. D. C. D. C. D. C. D. C. D. C. D. C. D. C. D. C. D. C. D. C. D. C. D. C. D. C. D. C. D. C. D. C. D. C. D. C. D. C. D. C. D. C. D. C. D. C. D. C. D. C. D. C. D. C. D. C. D. C. D. C. D. C. D. C. D. C. D. C. D. C. D. C. D. C. D. C. D. C. D. C. D. C. D. C. D. C. D. C. D. C. D. C. D. C. D. C. D. C. D. C. D. C. D. C. D. C. D. C. D. C. D. C. D. C. D. C. D. C. D. C. D. C. D. C. D. C. D. C. D. C. D. C. D. C. D. C. D. C. D. C. D. C. D. C. D. C. D. C. D. C. D. C. D. C. D. C. D. C. D. C. D. C. D. C. D. C. D. C. D. C. D. C. D. C. D. C. D. C. D. C. D. C. D. C. D. C. D. C. D. C. D. C. D. C. D. C. D. C. D. C. D. C. D. C. D. C. D. C. D. C. D. C. D. C. D. C. D. C. D. C. D. C. D. C. D. C. D. C. D. C. D. C. D. C. D. C. D. C. D. C. D. C. D. C. D. C. D. C. D. C. D. C. D. C. D. C. D. C. D. C. D. C. D. C. D. C. D. C. D. C. D. C. D. C. D. C. D. C. D. C. D. C. D. C. D. C. D. C. D. C. D. C. D. C. D. C. D. C. D. C. D. C. D. C. D. C. D. C. D. C. D. C. D. C. D. C. D. C. D. C. D. C. D. C. D. C. D. C. D. C. D. C. D. C. D. C. D. C. D. C. D. C. D. C. D. C. D. C. D. C. D. C. D. C. D. C. D. C. D. C. D. C. D. C. D. C. D. C. D. C. D. C. D. C. D. C. D. C. D. C. D. C. D. C. D. C. D. C. D. C. D. C. D. C. D. C. D. C. D. C. D. C. D. C. D. C. D. C. D. C. D. C. D. C. D. C. D. C. D. C. D. C. D. C. D. C. D. C. D. C. D. C. D. C. D. C. D. C. D. C. D. C. D. C. D. C. D. C. D. C. D. C. D. C. D. C. D. C. D. C. D. C. D. C. D. C. D. C. D. C. D. C. D. C. D. C. D. C. D. C. D. C. D. C. D. C. D. C. D. C. D. C. D. C. D. C. D. C. D. C. D. C. D. C. D. C. D. C. D. C. D. C.-D. D. C. D. C. D. C. D. C. D. C.-D. D. C. D. C. D. C. D. C. D. C. D. D. C. D. D. C. D. C. D. D. C. D. D. C. D. D. D. D. D. D. D. D. D. D. D. D. D. D. D. D. D. D. D. D. D. D. D. D. D. D. D. D. D. D. D. D. D. D. D. D. D. D. D. D. D. D. D. D. D. D. D. D. D. D. D. D. D. D. D. D. D. D. D. D. D. D. D. D. D. D. D. D. D. D. D. D. D. D. D. D. D. D. D. D. D. D. D. D. D. D. D. D. D. D. D. D. D. D. D. D. D. D. D. D. D. D. D. D. D. D. D. D. D. D. D. D. D. D. D. D. D. D. D. D. D. D. D. D. D. D. D. D. D. D. D. D. D. D. D. D. D. D. D. D. D. D. D. D. D. D. D. D. D. D. D. D. D. D. D. D. D. D. D. D. D. D. D. D. D. D. D. D. D. D. D. D. D. D. D. D. D. D. D. D. D. D. D. D. D. D. D. D. D. D. D. D. D. D. D. D. D. D. D. D. D. D. D. D. D. D. D. D. D. D. D. D. D. D. D. D. D. D. D. D. D. D. D. D. D. D. D. D. D. D. D. D. D. D. D. D. D. D. D. D. D. D. D. D. D. D. D. D. D. D. D. D. D. D. D. D. D. D. D. D. D. D. D. D. D. D. D. D. D. D. D. D. D. D. D. D. D. D. D. D. D. D. D. D. D. D. D. D. D. D. D. D. D. D. D. D. D. D. D. D. D. D. D. D. D. D. D. D. D. D. D. D. D. D. D. D. D. D. D. D. D. D. D. D. D. D. D. D. D. D. D. D. D. D. D. D. D. D. D. D. D. D. D. D. D. D. D. D. D. D. D. D. D. D. D. D. D. D. D. D. D. D. D. D. D. D. D. D. D. D. D. D. D. D. D. D. D. D. D. D. D. D. D. D. D. D. D. D. D. D. D. D. D. D. D. D. D. D. D. D. D. D. D. D. D. D. D. D. D. D. D. D. D. D. D. D. D. D. D. D. D. D. D. D. D. D. D. D. D. D. D. D. D. D. D. D. D. D. D. D. D. D. D. D. D. D. D. D. D. D. D. D. D. D. D. D. D. D. D. D. D. D. D. D. D. D. D. D. D. D. D. D. D. D. D. D. D. D. D. D. D. D. D. D. D. D. D. D. D. D. D. D. D. D. D. D. D. D. D. D. D. D. D. D. D. D. D. D. D. D. D. D. D. D. D. D. D. D. D. D. D. D. D. D. D. D. D. D. D. D. D. D. D. D. D. D. D. D. D. D. D. D. D. D. D. D. D. D. D. D. D. D. D. D. D. D. D. D. D. D. D. D. D. D. D. D. D. D. D. D. D. D. D. D. D. D. D. D. D. D. D. D. D. D. D. D. D. D. D. D. D. D. D. D. D. D. D. D. D. D. D. D. D. D. D. D. D. D. D. D. D. D. D. D. D. D. D. D. D. D. D. D. D. D. D. D. D. D. D. D. D. D. D. D. D. D. D. D. D. D. D. D. D. D. D. D. D. D. D. D. D. D. D. D. D. D. D. D. D. D. D. D. D. D. D. D. D. D. D. D. D. D. D. D. D. D. D. D. D. D. D. D. D. D. D. D. D. D. D. D. D. D. D. D. D. D. D. D. D. D. D. D. D. D. D. D. D. D. D. D. D. D. D. D. D. D. D. D. D. D. D. D. D. D. D. D. D. D. D. D. D. D. D. D. D. D. D. D. D. D. D. D. D. D. D. D. D. D. D. D. D. D. D. D. D. D. D. D. D. D. D. D. D. D. D. D. D. D. D. D. D. C. D. C. D. C. D. C. D. C. D. C. D. C. D. C. D. C. D. C. D. C. D. C. D. C. D. C. D. C. D. C. D. C. D. C. D. C. D. C. D. C. D. C. D. C. D. C. D. C. D. C. D. C. D. C. D. C. D. C. D. C. D. C. D. C. D. C. D. C. D. C. D. C. D. C. D. C. D. C. D. C. D. C. D. C. D. C. D. C. D. C. D. C. D. C. D. C. D. C. D. C. D. C. D. C. D. C. D. C. D. C. D. C. D. C. D. C. D. C. D. C. D. C. D. C. D. C. D. C. D. C. D. C. D. C. D. C. D. C. D. C. D. C. D. C. D. C. D. C. D. C. D. C. D. C. D. C. D. C. D. C. D. C. D. C. D. C. D. C. D. C. D. C. D. C. D. C. D. C. D. C. D. C. D. C. D. C. D. C. D. C. D. C. D. C. D. C. D. C. D. C. D. C. D. C. D. C. D. C. D. C. D. C. D. C. D. C. D. C. D. C. D. C. D. C. D. C. D. C. D. C. D. C. D. C. D. C. D. C. D. C. D. C. D. C. D. C. D. C. D. C. D. C. D. C. D. C. D. C. D. C. D. C. D. C. D. C. D. C. D. C. D. C. D. C. D. C. D. C. D. C. D. C. D. C. D. C. D. C. D. C. D. C. D. C. D. C. D. C. D. C. D. C. D. C. D. C. D. C. D. C. D. C. D. C. D. C. D. C. D. C. D. C. D. C. D. C. D. C. D. C. D. C. D. C. D. C. D. C. D. C. D. C. D. C. D. C. D. C. D. C. D. C. D. C. D. C. D. C. D. C. D. C. D. C. D. C. D. C. D. C. D. C. D. C. D. C. D. C. D. C. D. C. D. C. D. C. D. C. D. C. D. C. D. C. D. C. D. C. D. C. D. C. D. C. D. C. D. C. D. C. D. C. D. C. D. C. D. C. D. C. D. C. D. C. D. C. D. C. D. C. D. C. D. C. D. C. D. C. D. C. D. C. D. C. D. C. D. C. D. C. D. C. D. C. D. C. D. C. D. C. D. C. D. C. D. C. D. C. D. C. D. C. D. C. D. C. D. C. D. C. D. C. D. C. D. C. D. C. D. C. D. C. D. C. D. C. D. C. D. C. D. C. D. C. D. C. D. C. D. C. D. C. D. C. D. C. D. C. D. C. D. C. D. C. D. C. D. C. D. C. D. C. D. C. D. C. D. C. D. C. D. C. D. C. D. C. D. C. D. C. D. C. D. C. D. C. D. C. D. C. D. C. D. C. D. C. D. C. D. C. D. C. D. C. D. C. D. C. D. C. D. C. D. C. D. C. D. C. D. C. D. C. D. C. D. C. D. C. D. C. D. C. D. C. D. C. D. C. D. C. D. C. D. C. D. C. D. C. D. C. D. C. D. C. D. C. D. C. D. C. D. C. D. C. D. C. D. C. D. C. D. C. D. C. D. C. D. C. D. C. D. C. D. C. D. C. D. C. D. C. D. C. D. C. D. C. D. C. D. C. D. C. D. C. D. C. D. C. D. C. D. C. D. C. D. C. D. C. D. C. D. C. D. C. D. C. D. C. D. C. D. C. D. C. D. C. D. C. D. C. D. C. D. C. D. C. D. C. D. C. D. C. D. C. D. C. D. C. D. C. D. C. D. C. D. C. D. C. D. C. D. C. D. C. D. C. D. C. D. C. D. C. D. C. D. C. D. C. D. C. D. C. D. C. D. C. D. C. D. C. D. C. D. C. D. C. D. C. D. C. D. C. D. C. D. C. D. C. D. C. D. C. D. C. D. C. D. C. D. C. D. C. D. C. D. C. D. C. D. C. D. C. D. C. D. C. D. C. D. C. D. C. D. C. D. C. D. C. D. C. D. C. D. C. D. C. D. C. D. C. D. C. D. C. D. C. D. C. D. C. D. C. D. C. D. C. D. C. D. C. D. C. D. C. D. C. D. C. D. C. D. C. D. C. D. C. D. C. D. C. D. C. D. C. D. C. D. C. D. C. D. C. D. C. D. C. D. C. D. C. D. C. D. C. D. C. D. C. D. C. D. C. D. C. D. C. D. C. D. C. D. C. D. C. D. C. D. C. D. C. D. C. D. C. D. C. D. C. D. C. D. C. D. C. D. C. D. C. D. C. D. C. D. C. D. C. D. C. D. C. D. C. D. C. D. C. D. C. D. C. D. C. D. C. D. C. D. C. D. C. D. C. D. C. D. C. D. C. D. C. D. C. D. C. D. C. D. C. D. C. D. C. D. C. D. C. D. C. D. C. D. C. D. C. D. C. D. C. D. C. D. C. D. C. D. C. D. C. D. C. D. C. D. C. D. C. D. C. D. C. D. C. D. C. D. C. D. C. D. C. D. C. D. C. D. C. D. C. D. C. D. C. D. C. D. C. D. C. D. C. D. C. D. C. D. C. D. C. D. C. D. C. D. C. D. C. D. C. D. C. D. C. D. C. D. C. D. C. D. C. D. C. D. C. D. C. D. C. D. C. D. C. D. C. D. C. D. C. D. C. D. C. D. C. D. C. D. C. D. C. D. C. D. C. D. C. D. C. D. C. D. C. D. C. D. C. D. C. D. C. D. C. D. C. D. C. D. C. D. C. D. C. D. C. D. C. D. C. D. C. D. C. D. C. D. C. D. C. D. C. D. C. D. C. D. C. D. C. D. C. D. C. D. C. D. C. D. C. D. C. D. C. D. C. D. C. D. C. D. C. D. C. D. C. D. C. D. C. D. C. D. C. D. C. D. C. D. C. D. C. D. C. D. C. D. C. D. C. D. C. D. C. D. C. D. C. D. C. D. C. D. C. D. C. D. C. D. C. D. C. D. C. D. C. D. C. D. C. D. C. D. C. D. C. D. C. D. C. D. C. D. C. D. C. D. C. D. C. D. C. D. C. D. C. D. C. D. C. D. C. D. C. D. C. D. C. D. C. D. C. D. C. D. C. D. C. D. C. D. C. D. C. D. C. D. C. D. C. D. C. D. C. D. C. D. C. D. C. D. C. D. C. D. C. D. C. D. C. D. C. D. C. D. C. D. C. D. C. D. C. D. C. D. C. D. C. D. C. D. C. D. C. D. C. D. C. D. C. D. C. D. C. D. C. D. C. D. C. D. C. D. C. D. C. D. C. D. C. D. C. D. C. D. C. D. C. D. C. D. C. D. C. D. C. D. C. D. C. D. C. D. C. D. C. D. C. D. C. D. C. D. C. D. C. D. C. D. C. D. C. D. C. D. C. D. C. D. C. D. C. D. C. D. C. D. C. D. C. D. C. D. C. D. C. D. C. D. C. D. C. D. C. D. C. D. C. D. C. D. C. D. C. D. C. D. C. D. C. D. C. D. C. D. C. D. C. D. C. D. C. D. C. D. C. D. C. D. C. D. C. D. C. D. C. D. C. D. C. D. C. D. C. D. C. D. C. D. C. D. C. D. C. D. C. D. C. D. C. D. C. D. C. D. C. D. C. D. C. D. C. D. C. D. C. D. C. D. C. D. C. D. C. D. C. D. C. D. C. D. C. D. C. D. C. D. C. D. C. D. C. D. C. D. C. D. C. D. C. D. C. D. C. D. C. D. C. D. C. D. C. D. C. D. C. D. C. D. C. D. C. D. C. D. C. D. C. D. C. D. C. D. C. D. C. D. C. D. C. D. C. D. C. D. C. D. C. D. C. D. C. D. C. D. C. D. C. D. C. D. C. D. C. D. C. D. C. D. C. D. C. D. C. D. C. D. C. D. C. D. C. D. C. D. C. D. C. D. C. D. C. D. C. D. C. D. C. D. C. D. C. D. C. D. C. D. C. D. C. D. C. D. C. D. C. D. C. D. C. D. C. D. C. D. C. D. C. D. C. D. C. D. C. D. C. D. C. D. C. D. C. D. C. D. C. D. C. D. C. D. C. D. C. D. C. D. C. D. C. D. C. D. C. D. C. D. C. D. C.-D... Reduced Order Modeling Methods for Turbomachinery Design. *Doctoral dissertation, The Ohio State University*, 2008.
    \bibitem{SS6} S. Chakraborty, S. Balasubramanian, and S. Arun-Kumar. Data-Driven Surrogate Modeling Approaches for Parametric Prediction and Uncertainty Quantification of Fluid Flows. *AIAA Scitech 2023 Forum*, 2023.
    \bibitem{SS8} Y. Zerpa, J. S. K. S. S. K. S. K. S. K. S. K. S. K. S. K. S. K. S. K. S. K. S. K. S. K. S. K. S. K. S. K. S. K. S. K. S. K. S. K. S. K. S. K. S. K. S. K. S. K. S. K. S. K. S. K. S. K. S. K. S. K. S. K. S. K. S. K. S. K. S. K. S. K. S. K. S. K. S. K. S. K. S. K. S. K. S. K. S. K. S. K. S. K. S. K. S. K. S. K. S. K. S. K. S. K. S. K. S. K. S. K. S. K. S. K. S. K. S. K. S. K. S. K. S. K. S. K. S. K. S. K. S. K. S. K. S. K. S. K. S. K. S. K. S. K. S. K. S. K. S. K. S. K. S. K. S. K. S. K. S. K. S. K. S. K. S. K. S. K. S. K. S. K. S. K. S. K. S. K. S. K. S. K. S. K. S. K. S. K. S. K. S. K. S. K. S. K. S. K. S. K. S. K. S. K. S. K. S. K. S. K. S. K. S. K. S. K. S. K. S. K. S. K. S. K. S. K. S. K. S. K. S. K. S. K. S. K. S. K. S. K. S. K. S. K. S. K. S. K. S. K. S. K. S. K. S. K. S. K. S. K. S. K. S. K. S. K. S. K. S. K. S. K. S. K. S. K. S. K. S. K. S. K. S. K. S. K. S. K. S. K. S. K. S. K. S. K. S. K. S. K. S. K. S. K. S. K. S. K. S. K. S. K. S. K. S. K. S. K. S. K. S. K. S. K. S. K. S. K. S. K. S. K. S. K. S. K. S. K. S. K. S. K. S. K. S. K. S. K. S. K. S. K. S. K. S. K. S. K. S. K. S. K. S. K. S. K. S. K. S. K. S. K. S. K. S. K. S. K. S. K. S. K. S. K. S. K. S. K. S. K. S. K. S. K. S. K. S. K. S. K. S. K. S. K. S. K. S. K. S. K. S. K. S. K. S. K. S. K. S. K. S. K. S. K. S. K. S. K. S. K. S. K. S. K. S. K. S. K. S. K. S. K. S. K. S. K. S. K. S. K. S. K. S. K. S. K. S. K. S. K. S. K. S. K. S. K. S. K. S. K. S. K. S. K. S. K. S. K. S. K. S. K. S. K. S. K. S. K. S. K. S. K. S. K. S. K. S. K. S. K. S. K. S. K. S. K. S. K. S. K. S. K. S. K. S. K. S. K. S. K. S. K. S. K. S. K. S. K. S. K. S. K. S. K. S. K. S. K. S. K. S. K. S. K. S. K. S. K. S. K. S. K. S. K. S. K. S. K. S. K. S. K. S. K. S. K. S. K. S. K. S. K. S. K. S. K. S. K. S. K. S. K. S. K. S. K. S. K. S. K. S. K. S. K. S. K. S. K. S. K. S. K. S. K. S. K. S. K. S. K. S. K. S. K. S. K. S. K. S. K. S. K. S. K. S. K. S. K. S. K. S. K. S. K. S. K. S. K. S. K. S. K. S. K. S. K. S. K. S. K. S. K. S. K. S. K. S. K. S. K. S. K. S. K. S. K. S. K. S. K. S. K. S. K. S. K. S. K. S. K. S. K. S. K. S. K. S. K. S. K. S. K. S. K. S. K. S. K. S. K. S. K. S. K. S. K. S. K. S. K. S. K. S. K. S. K. S. K. S. K. S. K. S. K. S. K. S. K. S. K. S. K. S. K. S. K. S. K. S. K. S. K. S. K. S. K. S. K. S. K. S. K. S. K. S. K. S. K. S. K. S. K. S. K. S. K. S. K. S. K. S. K. S. K. S. K. S. K. S. K. S. K. S. K. S. K. S. K. S. K. S. K. S. K. S. K. S. K. S. K. S. K. S. K. S. K. S. K. S. K. S. K. S. K. S. K. S. K. S. K. S. K. S. K. S. K. S. K. S. K. S. K. S. K. S. K. S. K. S. K. S. K. S. K. S. K. S. K. S. K. S. K. S. K. S. K. S. K. S. K. S. K. S. K. S. K. S. K. S. K. S. K. S. K. S. K. S. K. S. K. S. K. S. K. S. K. S. K. S. K. S. K. S. K. S. K. S. K. S. K. S. K. S. K. S. K. S. K. S. K. S. K. S. K. S. K. S. K. S. K. S. K. S. K. S. K. S. K. S. K. S. K. S. K. S. K. S. K. S. K. S. K. S. K. S. K. S. K. S. K. S. K. S. K. S. K. S. K. S. K. S. K. S. K. S. K. S. K. S. K. S. K. S. K. S. K. S. K. S. K. S. K. S. K. S. K. S. K. S. K. S. K. S. K. S. K. S. K. S. K. S. K. S. K. S. K. S. K. S. K. S. K. S. K. S. K. S. K. S. K. S. K. S. K. S. K. S. K. S. K. S. K. S. K. S. K. S. K. S. K. S. K. S. K. S. K. S. K. S. K. S. K. S. K. S. K. S. K. S. K. S. K. S. K. S. K. S. K. S. K. S. K. S. K. S. K. S. K. S. K. S. K. S. K. S. K. S. K. S. K. S. K. S. K. S. K. S. K. S. K. S. K. S. K. S. K. S. K. S. K. S. K. S. K. S. K. S. K. S. K. S. K. S. K. S. K. S. K. S. K. S. K. S. K. S. K. S. K. S. K. S. K. S. K. S. K. S. K. S. K. S. K. S. K. S. K. S. K. S. K. S. K. S. K. S. K. S. K. S. K. S. K. S. K. S. K. S. K. S. K. S. K. S. K. S. K. S. K. S. K. S. K. S. K. S. K. S. K. S. K. S. K. S. K. S. K. S. K. S. K. S. K. S. K. S. K. S. K. S. K. S. K. S. K. S. K. S. K. S. K. S. K. S. K. S. K. S. K. S. K. S. K. S. K. S. K. S. K. S. K. S. K. S. K. S. K. S. K. S. K. S. K. S. K. S. K. S. K. S. K. S. K. S. K. S. K. S. K. S. K. S. K. S. K. S. K. S. K. S. K. S. K. S. K. S. K. S. K. S. K. S. K. S. K. S. K. S. K. S. K. S. K. S. K. S. K. S. K. S. K. S. K. S. K. S. K. S. K. S. K. S. K. S. K. S. K. S. K. S. K. S. K. S. K. S. K. S. K. S. K. S. K. S. K. S. K. S. K. S. K. S. K. S. K. S. K. S. K. S. K. S. K. S. K. S. K. S. K. S. K. S. K. S. K. S. K. S. K. S. K. S. K. S. K. S. K. S. K. S. K. S. K. S. K. S. K. S. K. S. K. S. K. S. K. S. K. S. K. S. K. S. K. S. K. S. K. S. K. S. K. S. K. S. K. S. K. S. K. S. K. S. K. S. K. S. K. S. K. S. K. S. K. S. K. S. K. S. K. S. K. S. K. S. K. S. K. S. K. S. K. S. K. S. K. S. K. S. K. S. K. S. K. S. K. S. K. S. K. S. K. S. K. S. K. S. K. S. K. S. K. S. K. S. K. S. K. S. K. S. K. S. K. S. K. S. K. S. K. S. K. S. K. S. K. S. K. S. K. S. K. S. K. S. K. S. K. S. K. S. K. S. K. S. K. S. K. S. K. S. K. S. K. S. K. S. K. S. K. S. K. S. K. S. K. S. K. S. K. S. K. S. K. S. K. S. K. S. K. S. K. S. K. S. K. S. K. S. K. S. K. S. K. S. K. S. K. S. K. S. K. S. K. S. K. S. K. S. K. S. K. S. K. S. K. S. K. S. K. S. K. S. K. S. K. S. K. S. K. S. K. S. K. S. K. S. K. S. K. S. K. S. K. S. K. S. K. S. K. S. K. S. K. S. K. S. K. S. K. S. K. S. K. S. K. S. K. S. K. S. K. S. K. S. K. S. K. S. K. S. K. S. K. S. K. S. K. S. K. S. K. S. K. S. K. S. K. S. K. S. K. S. K. S. K. S. K. S. K. S. K. S. K. S. K. S. K. S. K. S. K. S. K. S. K. S. K. S. K. S. K. S. K. S. K. S. K. S. K. S. K. S. K. S. K. S. K. S. K. S. K. S. K. S. K. S. K. S. K. S. K. S. K. S. K. S. K. S. K. S. K. S. K. S. K. S. K. S. K. S. K. S. K. S. K. S. K. S. K. S. K. S. K. S. K. S. K. S. K. S. K. S. K. S. K. S. K. S. K. S. K. S. K. S. K. S. K. S. K. S. K. S. K. S. K. S. K. S. K. S. K. S. K. S. K. S. K. S. K. S. K. S. K. S. K. S. K. S. K. S. K. S. K. S. K. S. K. S. K. S. K. S. K. S. K. S. K. S. K. S. K. S. K. S. K. S. K. S. K. S. K. S. K. S. K. S. K. S. K. S. K. S. K. S. K. S. K. S. K. S. K. S. K. S. K. S. K. S. K. S. K. S. K. S. K. S. K. S. K. S. K. S. K. S. K. S. K. S. K. S. K. S. K. S. K. S. K. S. K. S. K. S. K. S. K. S. K. S. K. S. K. S. K. S. K. S. K. S. K. S. K. S. K. S. K. S. K. S. K. S. K. S. K. S. K. S. K. S. K. S. K. S. K. S. K. S. K. S. K. S. K. S. K. S. K. S. K. S. K. S. K. S. K. S. K. S. K. S. K. S. K. S. K. S. K. S. K. S. K. S. K. S. K. S. K. S. K. S. K. S. K. S. K. S. K. S. K. S. K. S. K. S. K. S. K. S. K. S. K. S. K. S. K. S. K. S. K. S. K. S. K. S. K. S. K. S. K. S. K. S. K. S. K. S. K. S. K. S. K. S. K. S. K. S. K. S. K. S. K. S. K. S. K. S. K. S. K. S. K. S. K. S. K. S. K. S. K. S. K. S. K. S. K. S. K. S. K. S. K. S. K. S. K. S. K. S. K. S. K. S. K. S. K. S. K. S. K. S. K. S. K. S. K. S. K. S. K. S. K. S. K. S. K. S. K. S. K. S. K. S. K. S. K. S. K. S. K. S. K. S. K. S. K. S. K. S. K. S. K. S. K. S. K. S. K. S. K. S. K. S. K. S. K. S. K. S. K. S. K. S. K. S. K. S. K. S. K. S. K. S. K. S. K. S. K. S. K. S. K. S. K. S. K. S. K. S. K. S. K. S. K. S. K. S. K. S. K. S. K. S. K. S. K. S. K. S. K. S. K. S. K. S. K. S. K. S. K. S. K. S. K. S. K. S. K. S. K. S. K. S. K. S. K. S. K. S. K. S. K. S. K. S. K. S. K. S. K. S. K. S. K. S. K. S. K. S. K. S. K. S. K. S. K. S. K. S. K. S. K. S. K. S. K. S. K. S. K. S. K. S. K. S. K. S. K. S. K. S. K. S. K. S. K. S. K. S. K. S. K. S. K. S. K. S. K. S. K. S. K. S. K. S. K. S. K. S. K. S. K. S. K. S. K. S. K. S. K. S. K. S. K. S. K. S. K. S. K. S. K. S. K. S. K. S. K. S. K. S. K. S. K. S. K. S. K. S. K. S. K. S. K. S. K. S. K. S. K. S. K. S. K. S. K. S. K. S. K. S. K. S. K. S. K. S. K. S. K. S. K. S. K. S. K. S. K. S. K. S. K. S. K. S. K. S. K. S. K. S. K. S. K. S. K. S. K. S. K. S. K. S. K. S. K. S. K. S. K. S. K. S. K. S. K. S. K. S. K. S. K. S. K. S. K. S. K. S. K. S. K. S. K. S. K. S. K. S. K. S. K. S. K. S. K. S. K. S. K. S. K. S. K. S. K. S. K. S. K. S. K. S. K. S. K. S. K. S. K. S. K. S. K. S. K. S. K. S. K. S. K. S. K. S. K. S. K. S. K. S. K. S. K. S. K. S. K. S. K. S. K. S. K. S. K. S. K. S. K. S. K. S. K. S. K. S. K. S. K. S. K. S. K. S. K. S. K. S. K. S. K. S. K. S. K. S. K. S. K. S. K. S. K. S. K. S. K. S. K. S. K. S. K. S. K. S. K. S. K. S. K. S. K. S. K. S. K. S. K. S. K. S. K. S. K. S. K. S. K. S. K. S. K. S. K. S. K. S. K. S. K. S. K. S. K. S. K. S. K. S. K. S. K. S. K. S. K. S. K. S. K. S. K. S. K. S. K. S. K. S. K. S. K. S. K. S. K. S. K. S. K. S. K. S. K. S. K. S. K. S. K. S. K. S. K. S. K. S. K. S. K. S. K. S. K. S. K. S. K. S. K. S. K. S. K. S. K. S. K. S. K. S. K. S. K. S. K. S. K. S. K. S. K. S. K. S. K. S. K. S. K. S. K. S. K. S. K. S. K. S. K. S. K. S. K. S. K. S. K. S. K. S. K. S. K. S. K. S. K. S. K. S. K. S. K. S. K. S. K. S. K. S. K. S. K. S. K. S. K. S. K. S. K. S. K. S. K. S. K. S. K. S. K. S. K. S. K. S. K. S. K. S. K. S. K. S. K. S. K. S. K. S. K. S. K. S. K. S. K. S. K. S. K. S. K. S. K. S. K. S. K. S. K. S. K. S. K. S. K. S. K. S. K. S. K. S. K. S. K. S. K. S. K. S. K. S. K. S. K. S. K. S. K. S. K. S. K. S. K. S. K. S. K. S. K. S. K. S. K. S. K. S. K. S. K. S. K. S. K. S. K. S. K. S. K. S. K. S. K. S. K. S. K. S. K. S. K. S. K. S. K. S. K. S. K. S. K. S. K. S. K. S. K. S. K. S. K. S. K. S. K. S. K. S. K. S. K. S. K. S. K. S. K. S. K. S. K. S. K. S. K. S. K. S. K. S. K. S. K. S. K. S. K. S. K. S. K. S. K. S. K. S. K. S. K. S. K. S. K. S. K. S. K. S. K. S. K. S. K. S. K. S. K. S. K. S. K. S. K. S. K. S. K. S. K. S. K. S. K. S. K. S. K. S. K. S. K. S. K. S. K. S. K. S. K. S. K. S. K. S. K. S. K. S. K. S. K. S. K. S. K. S. K. S. K. S. K. S. K. S. K. S. K. S. K. S. K. S. K. S. K. S. K. S. K. S. K. S. K. S. K. S. K. S. K. S. K. S. K. S. K. S. K. S. K. S. K. S. K. S. K. S. K. S. K. S. K. S. K. S. K. S. K. S. K. S. K. S. K. S. K. S. K. S. K. S. K. S. K. S. K. S. K. S. K. S. K. S. K. S. K. S. K. S. K. S. K. S. K. S. K. S. K. S. K. S. K. S. K. S. K. S. K. S. K. S. K. S. K. S. K. S. K. S. K. S. K. S. K. S. K. S. K. S. K. S. K. S. K. S. K. S. K. S. K. S. K. S. K. S. K. S. K. S. K. S. K. S. K. S. K. S. K. S. K. S. K. S. K. S. K. S. K. S. K. S. K. S. K. S. K. S. K. S. K. S. K. S. K. S. K. S. K. S. K. S. K. S. K. S. K. S. K. S. K. S. K. S. K. S. K. S. K. S. K. S. K. S. K. S. K. S. K. S. K. S. K. S. K. S. K. S. K. S. K. S. K. S. K. S. K. S. K. S. K. S. K. S. K. S. K. S. K. S. K. S. K. S. K. S. K. S. K. S. K. S. K. S. K. S. K. S. K. S. K. S. K. S. K. S. K. S. K. S. K. S. K. S. K. S. K. S. K. S. K. S. K. S. K. S. K. S. K. S. K. S. K. S. K. S. K. S. K. S. K. S. K. S. K. S. K. S. K. S. K. S. K. S. K. S. K. S. K. S. K. S. K. S. K. S. K. S. K. S. K. S. K. S. K. S. K. S. K. S. K. S. K. S. K. S. K. S. K. S. K. S. K. S. K. S. K. S. K. S. K. S. K. S. K. S. K. S. K. S. K. S. K. S. K. S. K. S. K. S. K. S. K. S. K. S. K. S. K. S. K. S. K. S. K. S. K. S. K. S. K. S. K. S. K. S. K. S. K. S. K. S. K. S. K. S. K. S. K. S. K. S. K. S. K. S. K. S. K. S. K. S. K. S. K. S. K. S. K. S. K. S. K. S. K. S. K. S. K. S. K. S. K. S. K. S. K. S. K. S. K. S. K. S. K. S. K. S. K. S. K. S. K. S. K. S. K. S. K. S. K. S. K. S. K. S. K. S. K. S. K. S. K. S. K. S. K. S. K. S. K. S. K. S. K. S. K. S. K. S. K. S. K. S. K. S. K. S. K. S. K. S. K. S. K. S. K. S. K. S. K. S. K. S. K. S. K. S. K. S. K. S. K. S. K. S. K. S. K. S. K. S. K. S. K. S. K. S. K. S. K. S. K. S. K. S. K. S. K. S. K. S. K. S. K. S. K. S. K. S. K. S. K. S. K. S. K. S. K. S. K. S. K. S. K. S. K. S. K. S. K. S. K. S. K. S. K. S. K. S. K. S. K. S. K. S. K. S. K. S. K. S. K. S. K. S. K. S. K. S. K. S. K. S. K. S. K. S. K. S. K. S. K. S. K. S. K. S. K. S. K. S. K. S. K. S. K. S. K. S. K. S. K. S. K. S. K. S. K. S. K. S. K. S. K. S. K. S. K. S. K. S. K. S. K. S. K. S. K. S. K. S. K. S. K. S. K. S. K. S. K. S. K. S. K. S. K. S. K. S. K. S. K. S. K. S. K. S. K. S. K. S. K. S. K. S. K. S. K. S. K. S. K. S. K. S. K. S. K. S. K. S. K. S. K. S. K. S. K. S. K. S. K. S. K. S. K. S. K. S. K. S. K. S. K. S. K. S. K. S. K. S. K. S. K. S. K. S. K. S. K. S. K. S. K. S. K. S. K. S. K. S. K. S. K. S. K. S. K. S. K. S. K. S. K. S. K. S. K. S. K. S. K. S. K. S. K. S. K. S. K. S. K. S. K. S. K. S. K. S. K. S. K. S. K. S