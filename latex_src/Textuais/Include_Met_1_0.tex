%% =============================
%%      IMPORTANTE
%% ESTE ARQUIVO DEVE ESTAR SALVO COMO
%%      UTF - 8
%% =============================

% ----------------------------------------------------------
% Este capítulo é parte integrante do arquivo mestre
% Relatorio_TCC_Mestrado_Base_VERSÃO_SUBVERSÃO_FHZ
% ----------------------------------------------------------


% ----------------------------------------------------------
\chapter{methodology}
\label{chap:methodology}
% ----------------------------------------------------------

\section{Numerical Methods}

The present flow reconstrudtion method is purely data-driven and should work seamsly with experimental data. For the sake of cost and time, in this work, the data was syntheticly generated using numerical solvers. 


\section{POD}

POD, also known as principal component analysis (PCA), empirical orthogonal functions (EOF), the Hotelling transform, or a Karhunen-Loeve expansion in various domains, is a technique to reduce the dimensionality of complex data sets. This method is extensively used in numerical simulations and physical experiments to extract dominant patterns of variability from a high-dimensional dataset.